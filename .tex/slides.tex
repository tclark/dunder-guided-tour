\documentclass[aspectratio=169]{beamer}
\usetheme{default}
\usepackage{graphicx}
\usepackage{fancyvrb}
\usepackage{hyperref}

\newcommand\codeHighlight[1]{\textcolor[rgb]{1,0,0}{\textbf{#1}}}

\title{A Guided Tour of Dunder Methods}

\author{Tom Clark}
\date{}
\begin{document}


%----------- titlepage ----------------------------------------------%
\begin{frame}[plain]
  \titlepage
  
  \centerline{Before we start, you will want to clone the repository at}
  \centerline{ \url{https://github.com/tclark/dunder-guided-tour}.}
\end{frame}

%----------- frame ----------------------------------------------%
\begin{frame}
  \frametitle{Python!}
  
  There is plenty to love about Python, but in particular I love its clean and expressive syntax.
  \bigbreak
  
  Let's take a moment to appreciate how cool Python's strings are.
  
  
  \end{frame}

%----------- frame ----------------------------------------------%
\begin{frame}[fragile]
  \frametitle{Strings are great}
  
  \begin{verbatim}
  >>> s1 = "Spam"
  >>> s2 = "Cheese"
  \end{verbatim}  
  We can compare them
  \begin{verbatim}
  >>> s1 == s2
  False
  >>> s1 < s2
  False
  \end{verbatim}  
  and even do some arithmetic.
  \begin{verbatim}
  >>> s1 + s2
  'SpamCheese'
  >>> 3 * s1
  'SpamSpamSpam'
  \end{verbatim}
  
  \end{frame}

%----------- frame ----------------------------------------------%
\begin{frame}[fragile]
  \frametitle{Strings are great}
  
  \begin{verbatim}
  >>> s = "Spanish Inquisition"
  \end{verbatim}  
  Loops!
  \begin{verbatim}
  >>> for c in s:
  ...   print(c)
  ...
  S
  p
  a
  ...  
  \end{verbatim}
  
  \end{frame}


%----------- frame ----------------------------------------------%
\begin{frame}[fragile]
  \frametitle{Strings are great}
  
  \begin{verbatim}
  >>> s = "And now for something completely different"
  \end{verbatim}  
  We can access individual elements by index.
  \begin{verbatim}
  >>> s[4]
  'n'
  >>> s[-1]
  't'
  \end{verbatim}  
  We can slice them
  \begin{verbatim}
  >>> s[3:10:2]
  ' o o'
  \end{verbatim}
  
  \end{frame}

%----------- frame ----------------------------------------------%
\begin{frame}
  \frametitle{Dunder Methods}
  
  Strings and other built-in types get much of their power from the way
  they support operations like these, but it doesn't stop there.
  
  \bigbreak
  
  We can give our own classes the same expressive power by implementing the right
  \emph{dunder methods}.
  \end{frame}
  
%----------- frame ----------------------------------------------%
\begin{frame}[fragile]
  \frametitle{You've seen these}
  
  \begin{Verbatim}[commandchars=\\\{\}]
  
  class StorageJar:
      
      def \codeHighlight{__init__}(self, type):
          self.type = type
          
  
  jar = StorageJar("2 oz")         
  
  \end{Verbatim}
  \end{frame}
  
%----------- frame ----------------------------------------------%
\begin{frame}
  \frametitle{Warnings}
  
  \begin{itemize}
    \item You can create your own methods or attributes with double-underscored names. Don't do it.
    \item You can use an ``official" dunder method name but write the function so that its signature doesn't conform. Don't do it.
  \end{itemize}  
  
  Don't be reason we can't have nice things.
  \end{frame}
  
 %----------- frame ----------------------------------------------%
\begin{frame}
  \frametitle{The plan}
  
  We're going to look at some common Python features and see how our classes can work with them by implementing the necessary dunder methods
  
  \begin{itemize}
    \item String output
    \item Operator overloading
    \item Iterators
    \item Sequences
    \item Context managers
  \end{itemize}
  
  After we look at each one, we'll take a few minutes to do some hands-on coding examples before proceeding to the next.  
  \end{frame}
  
%----------- frame ----------------------------------------------%
\begin{frame}
  \frametitle{} 
  \centerline{\huge String output}
 \end{frame}
 
%----------- frame ----------------------------------------------%
\begin{frame}[fragile]
  \frametitle{Representing an object as a string}
  
  \begin{Verbatim}[commandchars=\\\{\}]
  >>> class Parrot:
  ...   def __init__(self, breed, alive):
  ...     self.breed = breed
  ...     self.alive = alive
  ...
  >>> p = Parrot("Norwegian Blue", True)
  >>> print(p)
  <__main__.Parrot object at 0x1012a8490>
  \end{Verbatim}
  
  \bigbreak
  \texttt{p} is of type \texttt{Parrot}, but the \texttt{print()} function needs a string. So what's happening?
  \end{frame}
  
 %----------- frame ----------------------------------------------%
\begin{frame}
  \frametitle{Obtaining a string from an object} 
  
   When we invoke \texttt{print(p)}
   
   \begin{enumerate}
     \item The interpreter checks whether the \texttt{Parrot} class implements \texttt{\_\_str\_\_()}. It doesn't.
     \item The interpreter falls back to using the \texttt{\_\_repr\_\_()} method. \texttt{Parrot} doesn't override it, so its parent implementation is used.
   \end{enumerate}  
 \end{frame}
  
%----------- frame ----------------------------------------------%
\begin{frame}[fragile]
  \frametitle{With dunders}
  
  \begin{Verbatim}
  >>> class Parrot:
  ...   def __init__(self, type, alive):
  ...     self.breed = breed
  ...     self.alive = alive
  ...   def __str__(self):
  ...     if self.alive:
  ...       return f'{self.breed} parrot'
  ...     return 'This is an ex-parrot!'
  ...   def __repr__(self):
  ...     return f"Parrot('{self.breed}', {self.alive})"
  ...
    \end{Verbatim}
 \end{frame}
  
  
 %----------- frame ----------------------------------------------%
\begin{frame}[fragile]
  \frametitle{With dunders}
  
  \begin{Verbatim}
  >>> p = Parrot('Norwegian Blue', True)
  >>> print(p)
  Norwegian Blue parrot
  >>> str(p)
  'Norwegian Blue parrot'
  >>> repr(p)
  "Parrot('Norwegian Blue', True)"  
  \end{Verbatim}
 \end{frame}
  
%----------- frame ----------------------------------------------%
\begin{frame}
  \frametitle{Guidelines} 
  
   
   
   \begin{enumerate}
     \item \texttt{\_\_str\_\_()} should return a user-friendly string.
     \item \texttt{\_\_repr\_\_()} should return a developer-friendly string. When practical, a string representing the call to the constructor that produces the object in preferred.
   \end{enumerate}  
 \end{frame}
  
%----------- frame ----------------------------------------------%
\begin{frame}
  \frametitle{} 
   
   \centerline{\huge Practice} 
   \centerline{See the \texttt{strings} directory in the tutorial repo.} 
   
   \end{frame}

%----------- frame ----------------------------------------------%
\begin{frame}
  \frametitle{} 
  \centerline{\huge Operator overloading}
 \end{frame}
 
%----------- frame ----------------------------------------------%
\begin{frame}[fragile]
  \frametitle{Recall what we know about strings}
  
  \begin{verbatim}
  >>> s1 = "Spam"
  >>> s2 = "Cheese"
  \end{verbatim}  
  We can compare them
  \begin{verbatim}
  >>> s1 == s2
  False
  >>> s1 < s2
  False
  \end{verbatim}  
  and even do some arithmetic.
  \begin{verbatim}
  >>> s1 + s2
  'SpamCheese'
  >>> 3 * s1
  'SpamSpamSpam'
  \end{verbatim}
  
  \end{frame}

%----------- frame ----------------------------------------------%
\begin{frame}
  \frametitle{} 
  
  Like many languages, Python supports a range operators that we typically associate with numbers. 
  Strings are not numbers, but for \emph{some} operators it's easy enough to imagine analogous operations so 
  Python overloads the operators.
  
  \bigbreak
  If you can imagine analogous operations for your own classes, you can implement operator overloading by defining the right dunder methods.
  
 \end{frame}
 
%----------- frame ----------------------------------------------%
\begin{frame}[fragile]
  \frametitle{Feeling a bit peckish}
  
  \begin{Verbatim}
  >>> class Cheese:
  ...   def __init__(self, name, runniness):
  ...     self.name = name
  ...     self.runniness = runniness
  ...     self.available = False  
  \end{Verbatim}
  
  \bigbreak
  Cheese is great, and we should be able to sort out what \texttt{==} means for cheeses.
 \end{frame}
  
%----------- frame ----------------------------------------------%
\begin{frame}[fragile]
  \frametitle{Cheesy equality}
  
  \begin{Verbatim}
  ...   def __eq__(self, other):
  ...     if isinstance(other, Cheese):
  ...       return (self.name == other.name 
  ...         and self.runniness == other.runniness)
  ...     return NotImplemented
  ...      
  \end{Verbatim}
  
  \bigbreak
  Two things to notice:
  \begin{itemize}
    \item We check the type of \texttt{other} to be sure we can handle the comparison.
    \item Rather than raising an exception, we return the special object \texttt{NotImplemented} if we can't make a comparison with other.
  \end{itemize}
 \end{frame}

%----------- frame ----------------------------------------------%
\begin{frame}[fragile]
  \frametitle{Comparing}
  
  \begin{Verbatim}
  >>> class Fruit:
  ...   def __init__(self, name):
  ...     self.name = name
  ...
  >>> edam = Cheese('Edam', 1)
  >>> cheddar = Cheese('Cheddar', 1)
  >>> apple = Fruit('Apple')
  >>> edam == cheddar
  False
  >>> edam == apple
  False
  >>> edam == edam
  True
  \end{Verbatim}
  
  \bigbreak
  Notice that no exception is raised, even when comparing fruit. This may or may not be a good thing.
 \end{frame}
 
\begin{frame}
  \frametitle{Comparison operators and their dunders}
  

\begin{tabular}{l l}
 
 ==    & \texttt{\_\_eq\_\_()} \\ 
 !=    & \texttt{\_\_ne\_\_()} \\ 
 $>$  & \texttt{\_\_gt\_\_()} \\ 
 $>$=    & \texttt{\_\_ge\_\_()} \\ 
 $<$   & \texttt{\_\_lt\_\_()} \\ 
  $<$=  & \texttt{\_\_le\_\_()} \\ 
\end{tabular}

Some notes:
\begin{itemize}
  \item You don't have to override any or all of these if they don't make sense for your classes.
  \item If you override ==, you get != ``for free", but best practise is to explicitly override !=.
  \item If you want to override all of these, see \texttt{functools.total\_ordering}.
\end{itemize}
\end{frame} 

%----------- frame ----------------------------------------------%
\begin{frame}[fragile]
  \frametitle{Arithmetic} 
  
  Suppose (bear with me) we wanted to add various \texttt{Cheese}s together to make some sort of Frankencheese? Again, we implement the 
  right dunder method.  
  
  \begin{verbatim}
  ...   def __add__(self, other):
  ...     if isinstance(other, Cheese):
  ...       newname = self.name + '/' + other.name
  ...       newrunny = (self.runniness + other.runniness) // 2  
  ...       return Cheese(newname, newrunny) 
  ...     return NotImplemented
  \end{verbatim}
   \end{frame}
 
%----------- frame ----------------------------------------------%
\begin{frame}[fragile]
  \frametitle{More arithmetic} 
  
  If we can add two \texttt{Cheese}s, then it also makes sense that we could multiply a \texttt{Cheese} and an integer.  For this slightly more complicated case we would need to 
  \bigbreak
  
  \begin{itemize}
    \item Implement \texttt{\_\_mul\_\_()}, for expressions like \texttt{colby * 3}. 
    \item Implement \texttt{\_\_rmul\_\_()}, for expressions like \texttt{3 * colby}.
    \item We might also want \texttt{\_\_imul\_\_()}, to support expressions like \texttt{colby *= 3}.
  \end{itemize}
  \bigbreak
  
  The actual implementations are left as an exercise (really). 
  
  \bigbreak
  See \url{https://docs.python.org/3.6/reference/datamodel.html?highlight=data model#emulating-numeric-types}
  \end{frame}
 
%----------- frame ----------------------------------------------%
\begin{frame}
  \frametitle{} 
   
   \centerline{\huge Practice} 
   \centerline{See the \texttt{operators} directory in the tutorial repo.} 
   
   \end{frame}

%----------- frame ----------------------------------------------%
\begin{frame}
  \frametitle{} 
  \centerline{\huge Iterators}
 \end{frame}

%----------- frame ----------------------------------------------%
\begin{frame}[fragile]
  \frametitle{Again, recall strings}
  
  \begin{verbatim}
  >>> s = "Spanish Inquisition" 
  >>> for c in s:
  ...   print(c)
  ...
  S
  p
  a
  ...  
  \end{verbatim}
  \bigbreak
  
  We say that Python strings are \emph{Iterables}, which means that they can supply 
  \emph{Iterators} that let us access their elements one at a time.
  
  \end{frame}

%----------- frame ----------------------------------------------%
\begin{frame}[fragile]
  \frametitle{Iterables/Iterators} 
  
    
  \begin{itemize}
    \item Iterables implement \texttt{\_\_iter\_\_()}, which returns an Iterator 
    \item Iterators implement \texttt{\_\_next\_\_()}, which returns successive elements from the iterable collection. When the elements have been exhausted, \texttt{\_\_next\_\_()} raises a \texttt{StopIteration} exception.
  \end{itemize}
 
    
    \bigbreak
    To be clear, these are two distinct but related objects\footnote{Usually}.
    \end{frame}

%----------- frame ----------------------------------------------%
\begin{frame}[fragile]
  \frametitle{Example: I'll have your spam}
  
  \begin{verbatim}
  >>> class Breakfast:
  ...   def __init__(self, items):
  ...     self.items = items
  ...
  >>> my_items = ['spam', 'spam', 'spam', 'spam', 'spam',
  ... 'spam', 'spam', 'baked beans', 'spam', 'spam', 'spam']
  >>> brekky = Breakfast(my_items)
  \end{verbatim}
  \bigbreak
  
  It would be nice to be able to iterate over all that spam.
  
  \end{frame}

%----------- frame ----------------------------------------------%
\begin{frame}[fragile]
  \frametitle{First approach}
  
  \begin{verbatim}
  ...   def __iter__(self):
  ...     return iter(self.items)
  ...
  \end{verbatim}
  \bigbreak
  
  \texttt{self.items} is a list, which is iterable. We just ask for its iterator.
 
  \end{frame}

%----------- frame ----------------------------------------------%
\begin{frame}[fragile]
  \frametitle{Second approach}
  
  \begin{verbatim}
  ...   def __iter__(self):
  ...     for item in self.items:
  ...       yield item
  ...
  \end{verbatim}
  \bigbreak
  
  This generator function returns a type of iterator
 
  \end{frame}

%----------- frame ----------------------------------------------%
\begin{frame}[fragile]
  \frametitle{Third approach}
  
  \begin{verbatim}
  ...   def __iter__(self):
  ...     return BreakfastIterator(self)
  ...
  \end{verbatim}
  \bigbreak
  
  For this we need to supply a \texttt{BreakfastIterator} class that implements
  \texttt{\_\_next\_\_()}
 
  \end{frame}

%----------- frame ----------------------------------------------%
\begin{frame}[fragile]
  \frametitle{End result}
  
  Whichever approach we use, we can now do this with our \texttt{Breakfast}.
  
  \begin{verbatim}
  >>> for food in brekky:
  ...   if food == 'spam':
  ...     eat(food)
  \end{verbatim}
   
  \end{frame}


%----------- frame ----------------------------------------------%
\begin{frame}
  \frametitle{} 
   
   \centerline{\huge Practice} 
   \centerline{See the \texttt{iterators} directory in the tutorial repo.} 
   
   \end{frame}

%----------- frame ----------------------------------------------%
\begin{frame}
  \frametitle{} 
  \centerline{\huge Sequences}
 \end{frame}

%----------- frame ----------------------------------------------%
\begin{frame}[fragile]
  \frametitle{Hey, let's consider strings one more time}
  
  \begin{verbatim}
  >>> s = "And now for something completely different"
  >>> s[4]
  'n'
  >>> s[-1]
  't'
  >>> s[3:10:2]
  ' o o'
  \end{verbatim}
  
  \bigbreak
  We can access a string's elements with an integer index or a slice because strings are \emph{sequences}.
  
  \end{frame}

%----------- frame ----------------------------------------------%
\begin{frame}[fragile]
  \frametitle{More spam}
  
  \begin{verbatim}
  >>> class Breakfast:
  ...   def __init__(self, items):
  ...     self.items = items
  ...
  >>> my_items = ['spam', 'spam', 'spam', 'spam', 'spam',
  ... 'spam', 'spam', 'baked beans', 'spam', 'spam', 'spam']
  >>> brekky = Breakfast(my_items)
  \end{verbatim}
  \bigbreak
  
  It makes some sense to be able to treat our \texttt{Breakfast} objects as sequences.
  
  \end{frame}
%----------- frame ----------------------------------------------%
\begin{frame}[fragile]
  \frametitle{First approach}
  
  \begin{verbatim}
  ...   def __getitem__(self, key):
  ...     return self.items[key]
  ...
  ...   def __len__(self):
  ...     return len(self.items)
  \end{verbatim}
  \bigbreak
  
  This works pretty well since \texttt{self.items} is a list, which is also a sequence, so we just delegate the work to the list.
  I do, however, have an issue with this.
 
  \end{frame}
%----------- frame ----------------------------------------------%
\begin{frame}[fragile]
  \frametitle{Second approach}
  
  \begin{verbatim}
  ...   def __getitem__(self, key):
  ...     if isinstance(key, slice):
  ...       return Breakfast(self.items[key])
  ...     return self.items[key]
  ...
  \end{verbatim}
  \bigbreak
  
  When we take a \emph{slice} of a \texttt{Breakfast}, the result should be also be a \texttt{Breakfast}.
  \end{frame}

%----------- frame ----------------------------------------------%
 \begin{frame}
   \frametitle{Mutable vs. immutable sequences}
   
   If our sequence is meant to be immutable, then we need only implement \texttt{\_\_getitem\_\_()}
   and \texttt{\_\_len\_\_()}.
   
   \bigbreak
   
   For mutable sequences we need to implement some other dunders. \texttt{Breakfast} is pretty mutable.
 \end{frame}  
 
%----------- frame ----------------------------------------------%
\begin{frame}[fragile]
  \frametitle{Methods for mutability}
  
  \begin{verbatim}
  ...   def __setitem__(self, key, val):
  ...     self.items[key] = val
  ...
  ...   def __delitem__(self, key):
  ...     del self.items[key]
  ...
  ...   def insert(self, key, val):
  ...     self.items.insert(key, val) 

  \end{verbatim}
  
  \bigbreak
  See \url{https://docs.python.org/3/library/collections.abc.html}
 \end{frame}

%----------- frame ----------------------------------------------%
\begin{frame}[fragile]
  \frametitle{The result}
  
  \begin{verbatim}
  >>> my_items = ['spam', 'spam', 'spam', 'spam', 'spam',
  ... 'spam', 'spam', 'baked beans', 'spam', 'spam', 'spam']
  >>> brekky = Breakfast(my_items)
  >>> brekky[2]
  'spam'
  >>> del brekky[7]
 \end{verbatim}
  
 \end{frame}


%----------- frame ----------------------------------------------%
\begin{frame}
  \frametitle{} 
   
   \centerline{\huge Practice} 
   \centerline{See the \texttt{sequences} directory in the tutorial repo.} 
   
   \end{frame}
%----------- frame ----------------------------------------------%
\begin{frame}
  \frametitle{} 
  \centerline{\huge Context managers}
 \end{frame}

%----------- frame ----------------------------------------------%
\begin{frame}[fragile]
  \frametitle{Oh, an example that's not a string.}
  
  \begin{verbatim}
  >>> with open("life_of_brian", "r") as script:
  ...   for line in script:
  ...     print(line)
  ...  
  \end{verbatim}
  
  \bigbreak 
  This is an elegant way to read a file. The thing that makes it work is a \emph{Context Manager},
  which is supplied by the call to \texttt{open()}.
  
  \end{frame}
%----------- frame ----------------------------------------------%
\begin{frame}[fragile]
  \frametitle{Required dunders}
  
  For a object to serve as a context manager, it needs to implement two dunders
  
  \bigbreak
  \texttt{\_\_enter\_\_()}, which prepares resources to be used in the code block defined by the \texttt{with} statement,
  and which is executed as the block is entered.
  
  \bigbreak
  \texttt{\_\_exit\_\_()}, which cleans up resources, and is executed when the code block is exited.

  \end{frame}
 
%----------- frame ----------------------------------------------%
\begin{frame}[fragile]
  \frametitle{Example context manager}
  
  \begin{verbatim}
  >>> class Announcer:
  ...   def __enter__(self):
  ...     print("It's...")
  ...
  ...   def __exit__(self, ex_type, ex_val, ex_trace):
  ...     print("And now for something completely different.")
  ...
  \end{verbatim}
  
    
  \end{frame}

%----------- frame ----------------------------------------------%
\begin{frame}[fragile]
  \frametitle{Example context manager}
  
  \begin{verbatim}
  >>> with Announcer():
  ...   print("Monty Python's Flying Circus")
  ...
  It's...
  Monty Python's Flying Circus
  And now for something completely different.  
  
  \end{verbatim}
  
    
  \end{frame}
 
%----------- frame ----------------------------------------------%
\begin{frame}
  \frametitle{} 
   
   \centerline{\huge Practice} 
   \centerline{See the \texttt{contexts} directory in the tutorial repo.} 
   
   \bigbreak
   \centerline{\huge Thank you!} 
   \end{frame}




\end{document}
